\documentclass[11pt]{article}
\usepackage{amssymb}
\usepackage{amsmath}
\usepackage{bbm}
\usepackage{centernot}
\usepackage{amsfonts}
\usepackage{eurosym}
\usepackage{geometry}
\usepackage{ulem}
\usepackage{graphicx}
\usepackage{tikz}
\usepackage{rotating}
\usepackage{caption}
\usepackage{color}
\usepackage{setspace}
\usepackage{sectsty}
\usepackage{comment}
\usepackage{footmisc}
\usepackage[inline]{enumitem}
\usepackage{caption}
\usepackage{natbib}
\usepackage{pdflscape}
\usepackage{subfigure}
\usepackage{array}
\usepackage{titling}
\usepackage{multirow}
\usepackage{diagbox}
\usepackage{dcolumn}
\usepackage{makecell}
\usepackage[hidelinks]{hyperref}
\hypersetup{unicode = true}

\normalem

\onehalfspacing
\newtheorem{theorem}{Theorem}
\newtheorem{corollary}[theorem]{Corollary}
\newtheorem{proposition}{Proposition}
\newenvironment{proof}[1][Proof]{\noindent\textbf{#1.}}{\ \rule{0.5em}{0.5em}}

\newtheorem{hyp}{Hypothesis}
\newtheorem{subhyp}{Hypothesis}[hyp]
\renewcommand{\thesubhyp}{\thehyp\alph{subhyp}}
% \renewcommand{\labelenumi}{H.\arabic{enumi}.} % Redefine new labels for hyp
\newcommand{\T}{\rule{0pt}{2.6ex}}            % Top strut
\newcommand{\B}{\rule[-1.2ex]{0pt}{0pt}}      % Bottom strut
\newcommand{\red}[1]{{\color{red} #1}}
\newcommand{\blue}[1]{{\color{blue} #1}}
\newcommand{\ci}{\perp\!\!\!\perp}
\newcommand{\nci}{\centernot{\ci}}
\newcommand{\subtitle}[1]{\posttitle{\par\end{center}\begin{center}\large#1\end{center}\vskip0.5em}}
\newcommand{\refp}[1]{(\ref{#1})}

\newcolumntype{L}[1]{>{\raggedright\let\newline\\arraybackslash\hspace{0pt}}m{#1}}
\newcolumntype{C}[1]{>{\centering\let\newline\\arraybackslash\hspace{0pt}}m{#1}}
\newcolumntype{R}[1]{>{\raggedleft\let\newline\\arraybackslash\hspace{0pt}}m{#1}}

\geometry{left=1.0in,right=1.0in,top=1.0in,bottom=1.0in}

\begin{document}

\begin{titlepage}
\title{Judicial Favoritism of Politicians: Evidence from Small Claims Courts}
\author{Andre Assumpcao\thanks{PhD Student, Department of Public Policy, The University of North Carolina at Chapel Hill. Contact details: \href{mailto:aassumpcao@unc.edu}{\textcolor{blue}{aassumpcao@unc.edu}}}}
\date{December 21, 2018}

\maketitle

\begin{abstract}
\noindent While there are many studies investigating judicial favoritism on the basis of race, ethnicity, or political connectedness, there is scant evidence on judicial collusion with other branches of government. In this paper, I claim that certain incentive structures might give rise to judicial favoritism and collusion across branches of government. I test this hypothesis by looking at rulings issued by the state court of São Paulo, Brazil, in small claims cases involving members of municipal executive and legislative bodies. Using an empirical strategy developed by \citet{AbramsJudgesVaryTheir2012} with Monte Carlo simulations of judicial bias and employing machine learning techniques, I respectively test pro-politician bias and make predictions of court outcomes for the entire distribution of elected politicians in São Paulo over a ten year period. To my knowledge, this is the first study providing evidence for power collusion in developing countries. \\
\vspace{0in} \\
\noindent\textbf{Keywords:} judicial politics; judicial bias; political economy of development; law and social science. \\

\noindent\textbf{JEL classification:} D73; K42; P48; H83. \\

\vspace{0in}
\bigskip

\end{abstract}

\setcounter{page}{0}

\thispagestyle{empty}

\end{titlepage}

\clearpage

\section{Introduction} \label{sec:introduction_paper2}

Suppose a case involving a politician is brought before an independent, high-quality court system; judges make their decision uniquely based on case merits. Assume further that lawyer skills and case merits are randomly distributed across plaintiffs and defendants. Under these simplifying conditions, politicians should not expect a higher win rate at trial than ordinary citizens as their power would be an irrelevant factor for the court. Surprisingly, however, there are not many studies measuring judicial independence in cases involving politicians. This paper is one of the first attempts at measuring judicial impartially and predicting court outcomes when politicians are before judges.

There is comprehensive evidence supporting other discrimination effects. \citet{ShayoJudicialIngroupBias2011} document a positive in-group bias, or preferential treatment, of 17 to 20 percentage points when judge and litigants have the same ethnicity. \citet{AbramsJudgesVaryTheir2012} find that African American defendants are 18 percentage points more likely to be incarcerated than white defendants. \citet{LuPoliticalConnectednessCourt2015} show that politically connected firms are more likely to have favorable rulings in property rights cases in China. A number of additional discrimination cases are reported in \citet{RachlinskiJudgingJudiciaryNumbers2017} and though there are multiple studies on the political ideology of judges, there are no studies on how supposedly independent judges behave when their equal branches of power are being challenged in court.

Isolating the effect of judicial favoritism on court outcomes is not easy. Court cases are filled with sources of heterogeneity. Judges, plaintiffs, and claimants have individual traits that could influence an outcome, such as their gender, ethnicity, religion, wealth, and so on. Litigants might also have access to heterogeneous pools of lawyers; any minimal discrepancy in skills might be the deciding factor driving the outcome of a case. Finally, case circumstances and merits change substantially and can determine how any single judge will rule. For instance, \citet{LimJudgePoliticianPress2015} evaluate whether judicial decisions are influenced by media coverage and find that nonpartisan U.S. State Court judges increase sentence length in violent crimes by 3.4 percent (equivalent to six months of extra jail time) when media coverage is high. The combination of these factors makes it hard for the isolation of a single bias effect on court outcomes.

In recent years, however, there has been growing interest in the relationship between judicial decisions and politics. Using data on employment claims filed in Venezuela between 2006 and 2017, \citet{Sanchez-MartinezDismantlingInstitutionsCourt2018} looks at whether defendant employers are more likely to see a favorable outcome when they are affiliated with the United Socialist Party of Venezuela (PSUV), in power since 2007. The author finds that employers who share the same party affiliation as judges are 20 percent more successful at trial. In \citet{LambaisJudicialSubversionEvidence2018}, the authors identify a 50 percentage point advantage in the win rate at trial for elected versus non-elected candidates when both are defendants in corruption cases filed after election day has passed. This project supplements the recent literature by investigating whether favoritism persists in cases where judges and politicians have less at stake (e.g.~small claims cases), testing a theory of power collusion across branches of government.

Besides the theoretical component, this paper makes a series of data analysis contributions to the literature in law and politics. In order to measure the effect of political bias, I scrape and code judicial decisions from the São Paulo State Court System (Brazil) for all elected mayors and city councilors since 2008. I then apply the methodology in \citet{AbramsJudgesVaryTheir2012} to evaluate whether politicians have an upper hand in small claims cases. I construct a random distribution of court outcomes against which I compare the observed outcomes in the data scraped from the web. Finally, I employ text analysis to extract additional information from judicial decisions and construct supervised machine learning (ML) predictions of court outcomes based on case and politician characteristics for the entire distribution of elected politicians in the state of São Paulo between 2008 and 2018. The web scraper, the simulations, and the ML algorithms are made available as free software for researchers conducting similar research projects using judiciary data in other countries.

The remainder of this proposal is as follows. Section \ref{sec:background_paper2} presents the institutional environment of Brazil's State Court Systems. Section \ref{sec:data_paper2} summarizes the test dataset used for analysis; the theoretical framework is presented in section \ref{sec:theory_paper2}; section \ref{sec:methods_paper2} discusses the analytical strategy and, finally, section \ref{sec:conclusion_paper2} lays out the necessary steps for completion of this project.

\section{Institutional Background} \label{sec:background_paper2}

Brazil's judiciary system is divided into general and limited jurisdiction courts. Federal and State Courts form the general system and Electoral, Military, and Labor Courts form the limited jurisdiction system. There are up to four instances of judicial review in either system and the court of last resort is the Federal Supreme Court (STF). It takes up cases under its jurisdiction as set out in the Brazilian Constitution, cases in which there are conflicting norms or jurisprudence issued by lower courts, and cases where there is a direct violation of constitutional norms. To limit the sources of heterogeneity, this paper focuses on cases heard at state court systems. In particular, I focus on the state of São Paulo, the most economically and politically important state in the country.

There are 319 judicial districts in the state and each district has one or more courthouses. These courthouses host at least one judge with either a broad mandate, meaning that they can rule on any issue within the state court system attributions, or a narrow mandate, which means they only oversee certain types of cases within the system, e.g.~commercial or family law. Within the state system, there are specialized small claims courts called \emph{Juízados Especiais Cíveis} (Special Civil Tribunals, in free translation, and SCTs henceforth). SCTs replaced the primary small claims courts across Brazil upon the passage of the latest Brazilian Constitution in 1988.\footnote{More evidence of this in \citet{LichandAccessJusticeEntrepreneurship2014}.} Their goal is to simplify and increase access to justice across states by means of removing many procedural requirements present in other litigation instruments. SCTs are the primary judicial body for small complexity cases, defined as cases in which claims do not exceed 40 times the minimum wage\footnote{There are no state minimum wages in Brazil, so this is the federal, nationwide minimum wage at R\$ 954.00 in 2018. This is equivalent to ~\$10,500 in current dollars using the 2018 exchange rate average} involving lease breach, consumer rights, debt executions, torts, and so on. There is no need for hiring an attorney if claims are under 20 times the minimum wage. SCTs are only open to individual or small company plaintiffs.

An example helps illustrate a typical SCT case. Suppose your mobile phone service provider has been overbilling you for international phone calls that were never made. You, unfortunately, could not resolve this issue with the company's customer service and now would like to take legal action and receive financial compensation for the wrongful charges to your bill. You walk up to an SCT office, speak to a courthouse clerk and file your claim along with any supporting documentation. The clerk then provides a court date for a conciliation hearing. At the hearing, you and the phone company will try to reach an agreement; if that fails, the judge sets trial for either later that same day or in the following days. At trial, the judge issues a sentence which can be appealed within 10 days; on appeal, a three-judge panel then issues the final ruling. This entire process might take less than three months, representing a substantial improvement when compared to cases in the regular judicial process at state courts.

SCT structure greatly reduces the number of dimensions driving judicial decisions. According to the São Paulo State Court website, there are less than 15 types of cases that can be brought before SCTs. It is then easier for judges and lawyers, when hired by the parts, to specialize and reduce any skill discrepancy that could substantially alter a case outcome. The sentence is also standard across cases: the losing side will pay the claim amount to the winning side, which is capped at 40x the minimum wage. The standard, and relatively low salience, punishment to litigants removes an additional source of heterogeneity from high-profile cases, such as corruption cases in \citet{LambaisJudicialSubversionEvidence2018} or violent crimes in \citet{LimJudgePoliticianPress2015}. In fact, the use of small claims court is an approach first introduced by \citet{ShayoJudicialIngroupBias2011}, that take advantage of the relative homogeneity of small claims cases in Israel to isolate the effect of ethnicity on court outcomes. Lastly, judges have no control over which SCT cases they take. In single-judge benches, all cases are presented before the same judge; in multiple-judge benches, the cases are randomly distributed to judges assigned to each SCT. These distribution rules are again dimension-reducing and prevent that cases are differently distributed to systematically more lenient (or harsher) judges at the state system.\footnote{Yet, for robustness purposes, I replicate the process in \citet{AbramsJudgesVaryTheir2012} producing random distributions of court outcomes to serve as a check on the quality of the case assignment system implemented by the state of São Paulo.}

\section{Data} \label{sec:data_paper2}

The data come from two sources. First, I use information on candidates running for municipal office in the state of São Paulo in 2008, 2012, and 2016 from the Brazilian Electoral Court (TSE). TSE has jurisdiction over the entire electoral process in Brazil, from registering candidates, ruling over breach of electoral law, and overseeing the voting process on election day, to counting votes and authorizing that elected politicians take up office. Though TSE is a permanent court, the busiest period in the electoral process occurs between August and December every two years, and politicians begin mandates on Jan 1\textsuperscript{st}.

In every two-year cycle, TSE produces a comprehensive compilation of electoral data. It collects individual-level data on politicians and publishes results by electoral section, which is the actual physical building where electronic voting machines are arranged on election day (there is no early neither mail-in voting in Brazil). I use TSE electoral results, candidate information, and electoral district data for every elected candidate in the state of São Paulo in the municipal elections of 2008, 2012, and 2016 elections. Table \ref{tab:sumstats} contains a sample of the variables for the universe of elected candidates in the state.


\begin{table}[!htbp] \centering
  \caption{Descriptive Statistics}
  \label{tab:sumstats}
\scriptsize
\begin{tabular}{@{\extracolsep{5pt}}lccccc}
\\[-1.8ex]\hline
\hline \\[-1.8ex]
& \multicolumn{1}{c}{N} & \multicolumn{1}{c}{Mean} & \multicolumn{1}{c}{St. Dev.} & \multicolumn{1}{c}{Min} & \multicolumn{1}{c}{Max} \T \B \\
\hline \\[-1.8ex]
Age                             & 15,232 & 45.384 & 10.638 & 18 & 89 \\
Male                            & 15,232 & .882   & .322   & 0  & 1  \\
Political Experience            & 15,232 & .139   & .346   & 0  & 1  \\
Campaign Expenditures (ln)      & 15,232 & 9.232  & 4.099  & 0  & 16 \\
% Politician is Plaintiff         & 15,232 & .504   & .500   & 0  & 1  \\
% Probability of Favorable Ruling & 15,232 & .509   & .500   & 0  & 1  \\
\\[-1.8ex]\hline
\hline \\[-1.8ex]
\end{tabular}
\end{table}


The average age of elected candidates in the state is 45 years old, 88.5 percent are male, and 38.8 percent have previous political experience, measured as an indicator variable for candidates who have been reelected or have declared their occupation in the TSE form as politician of any kind (city councilor, mayor, governor, member of Congress, senator, president). I have also collected categorical variables for educational attainment and marital status for all elected officials but they are omitted from the table. The most frequent educational level and marital status are four-year college degree or equivalent (36 percent) and married (70.9 percent), respectively.

In the full version of this paper, I search for SCT cases involving these elected politicians in the database of public judicial decisions maintained by the São Paulo State Court (TJ-SP). I predict about a quarter of these politicians were involved in small claims cases, so I expect a sample of about 5,000 cases involving politicians in the state. TJ-SP publishes all judicial decisions since 2008 on their website and the information available is the case date, type (breach of contract, debt execution, and others above), court where it has been filed, ruling judge, amount claimed, litigants and their lawyers (if hired), sentences, and every other case movement through the system. Using these documents, I can recover a rich set of information for each court case involving a politician and pinpoint potential factors driving a judge's decision.

Stepwise data collection is as follows. First, I use a politician's name to find the court cases they were involved. I match politicians to plaintiffs or defendants by setting a maximum Levenshtein distance across names after randomly sampling all cases matched and checking for errors. Second, I filter down case hits to SCT-only hits and scrape each case number, which are individual identifiers for every single court case in Brazil.\footnote{The National Council of Justice, the main judicial oversight body, passed legislation in 2006 (fully enacted in 2008), establishing a 20-digit individual identifier for all court cases in Brazil.} The case numbers are a more stable search criterion than name. Finally, I use these numbers to download all other case information. Though this process seems overwhelming, I have already started writing the programs necessary for scraping the TJ-SP website. In fact, a research collaborator in the Brazilian Association of Jurimetrics has functioning web scrapers written in R which download case data using different search criteria (case number and type, litigant name, court type and location, and so on). I am just translating the R code into Python for more speed and functionality.

\section{Theoretical Framework} \label{sec:theory_paper2}

Suppose there are three representative agents in Brazil, one for each branch of government: the executive, the legislative, and the judiciary. Though these agents are independent, they interact with one another over time. The executive agent serves on one or two four-year mandates (pending reelection at the end of the first term). They control the majority of government budgets and have the discretion to set wages and resources allocated to the other branches. The legislative agent serves on a four-year mandate,\footnote{Senators serve for eight years but this does not change the theoretical predictions in this section.} which is renewable as many times as they are reelected; they have no term limit. They are responsible for passing law and determining budget levels but not its composition. In other words, they approve the amount of money available for other branches of government but do not have a say on how the money will be spent. The judiciary agent serves on life-long mandates and yields power in restrictive but steady ways. They have limited control over resources as they only oversee budgets in the courts at which they serve, but resolve disputes between the other two branches of government and other economic agents (individuals, companies, etc). In this simplified model, the judiciary pleases or upsets the executive and legislative by settling their disputes.

I am interested in the behavior of the judiciary with respect to other branches of government. Power collusion could exist if the judiciary were using sentences to please the executive and legislative as a means of buying out those who make the calls on judiciary resources. Under this hypothesis, the representative judiciary agent derives utility in each period $t$ according to equation \refp{eq:u_jud}, which describes the benefit $f$ as a function of $k$ observable characteristics $\sum_{x = 1}^{k}x_{k}$, such as their time in post, their wages, their working conditions, and unobservable characteristics $\varepsilon$ such as reputation and their happiness in serving justice; and costs $c$, which are a function of $m_{t}$ working conditions, executive and legislative utilities $u_{e}^{t}$ and $u_{l}^{t}$, and $\delta_{t}$ are exogenous, stochastic shocks that impact judicial work. These per-period utilities are computed in perpetuity in accordance with judge's mandates: \\

\begin{equation} \label{eq:u_jud}
  u_{\text{j}}^{t} = \frac{1}{r} \times \sum_{t = 1}^{t} \Bigg[ f\Big(\sum_{x = 1}^{k}x_{k}, \ \varepsilon_{t} \Big) - c\Big(\sum_{m = 1}^{k} m_{t}, \ u_{e}^{t}(p), \ u_{l}^{t}(p), \ \delta_{t}\Big) \Bigg]
\end{equation} \\

Since the executive and the legislative agents have primary responsibility for government budget, we can thus expect the judiciary to strategically maximize future net benefit by pleasing other agents and thus reducing their negative impact on the judiciary utility. Indeed, this incentive is particularly strong in low salience cases, such as those filed in small claims courts. First, the penalties are limited to a small monetary amount privately held by the defendant. A ruling in SCT does not impact government budget (as if it were a corruption case in which the funds might be returned and used for public goods). In addition, these cases are not likely to cause any significant political damage, as they do not imply any government wrongdoing. Most defendant politicians are brought to court under tort allegations and these are perceived as natural in the political process, such that convictions have a weaker informational effect compared to other crimes (again, corruption). To that end, the judiciary agent manipulates the probability they hand out court outcomes involving members of the executive or the legislative branches, summarized by $p$ in their utilities. An increase in the probability of a favorable outcome for members of the executive and the legislative branches increases their per-period utility and reduces the cost for the judiciary. This relationship is summarized in equation \refp{eq:u_jud2} below: \\

\begin{equation} \label{eq:u_jud2}
  \cfrac{\partial u_{j}^{t}}{\partial p} = \cfrac{\partial u_{j}^{t}}{\partial c} \cdot \cfrac{\partial c}{\partial u_{e, \, l}^{t}} \cdot \cfrac{\partial u_{e, \, l}^{t}}{\partial p} > 0
\end{equation} \\

The sign in the first partial derivate in the right-hand side is straightforward, as one can expect that an increase in costs reduces the judiciary agent's wellbeing ($\partial u_{j}^{t} / \partial c < 0$). Since the cost to the judiciary is a function of the utilities of the executive and the legislative, and any increase in the latter two utilities favors the judiciary, I expect a negative relationship between $c$ and $u_{e}^{t}$ or $u_{l}^{t}$ such that the sign in the second derivate in the right-hand side of equation \refp{eq:u_jud2} is also negative ($\partial c / \partial u_{e, \, l}^{t} < 0$). Finally, a favorable court outcome directly increases one's utility, so the compounded expected result of the partial derivate of the judiciary's utility with respect to the probability of a favorable outcome in court is positive. By manipulating $p$, the judiciary agent is minimizing cost and maximizing the net (per-period) benefit of holding office. In the absence of the pro-politician bias, equation \refp{eq:u_jud} would not be conditional on the utilities of the executive and the legislative agents such that $\partial c / \partial u_{e, \, l}^{t} \cdot \partial u_{e, \, l}^{t} / \partial p = 0$ and, consequently, $\partial u_{j}^{t} / \partial p = 0$ in equation \refp{eq:u_jud2}.

\section{Empirical Strategy} \label{sec:methods_paper2}

This paper has two goals. First, I want to test the existence of judicial bias in favor of politicians when a case is presented to judges in SCT, the small claims court in Brazil. Second, I want to predict the court outcome for the entire distribution of politicians in the state of São Paulo, even if they have not been involved in SCT cases. These are two relevant empirical exercises for any country aiming at the improvement of its \emph{de facto} separation of power.

\subsection{Judicial Favoritism} \label{subsec:methods1_paper2}

For the judicial favoritism exercise, I build on the methodology developed by \citet{AbramsJudgesVaryTheir2012}, who analyze racial bias in felony cases sentencing across judges in Cook County, IL. The authors suggest a two-step process to measure sentencing bias, which is analogous to outcome bias in this project: (i) confirm random assignment of cases, such that outcomes are comparable across judges and not due to any excessive harshness or leniency uncorrelated with plaintiffs or defendants condition as politicians; (ii) test whether any heterogeneity in case outcomes involving politicians are not due to sampling variability driven by other factors (case merits, court conditions, time in which case was tried) or by chance. \citet{AbramsJudgesVaryTheir2012} suggest that (i) random assignment can be tested regressing a case characteristic (e.g.~gender) on multiple control variables, as below: \\

\begin{equation} \label{eq:methods1}
  \text{Female}_{ijc} = \alpha + \sum_{k=1}^{k} \beta x_{kijc} + \sum_{n=1}^{n} \Gamma D_{n} + \sum_{c=1}^{c} \lambda_{c} + \varepsilon_{ijc}
\end{equation} \\

Where Female$_{ijc}$ is a female binary indicator for the plaintiff's gender, $x_{k}$ are $k$ control variables, $D_{n}$ is a matrix of $n$ judge fixed-effects, $\lambda_{c}$ is a matrix of court fixed-effects, and subscripts $i$, $j$, $c$ are indexing plaintiff $i$, judge $j$, and court $c$. Under random assignment, the \emph{F}-test on the joint distribution of judge fixed-effects should fail to reject the null (i.e.~fixed-effects are equal). The bias assessment would follow from a similar specification where the dependent variable is the favorable outcome for politicians (Pro-Politician$_{ijc}$):

\begin{equation}
  \begin{split} \label{eq:methods2}
    \text{Pro-Politician}_{ijc} = \alpha & + \sum_{k=1}^{k} \beta x_{kijc} + \text{Politician}_{ijc} + \sum_{n=1}^{n} \Gamma D_{n} \\
    & + \sum_{n=1}^{n} \Gamma D_{n} \cdot \text{Politician}_{ijc} + \sum_{c=1}^{c} \lambda_{c} + \varepsilon_{ijc}
  \end{split}
\end{equation} \\

Politician is an indicator variable for elected office-holder plaintiffs at the time their case is ruled at court $c$. All other variables are the same as in equation \refp{eq:methods1} and a second \emph{F}-test on the interactions between judge-fixed effects and politician plaintiffs is a test on judicial bias. This design, however, suffers from the same problems as \citet{AbramsJudgesVaryTheir2012}, i.e.~the overrejection of the null in the \emph{F}-test for two reasons: (i) since the number of judges per SCT court is relatively small, either \emph{F}-test does not meet its asymptotic properties and suffers from finite-sample bias (the judge variability within courts is small); (ii) the conventional \emph{F}-statistic will overreject the null when errors are not normally distributed, as is the case with a binary outcome such as the pro-politician ruling in this project.

The solution to the overrejection problem is the construction of simulated datasets where assignment of cases is indeed random and the subsequent comparison of statistical moments in the empirical distribution versus the simulated moments. The process is as follows. First, the researcher should group the actual (empirical) sample into the randomization units, which are the many state SCTs in this case. Within these units, the researcher creates simulated observations, for each judge, from random draws (with replacement) of each of the variables in the empirical sample unit. Suppose there are 20 observed cases heard by four judges (five cases each) in a given SCT in the state of São Paulo. Each case has a set of observed characteristics, e.g.~plaintiff gender, whether plaintiff was a politician at the time the case was heard, claim amount, etc. The researcher then creates 20 simulated cases, five per judge (keeping the same proportion as in the original data), where each case characteristic is randomly drawn from the sample of 20 observed cases in the empirical dataset. Once this process is replicated for all randomization units (SCTs), a simulated dataset of the same size as the empirical dataset has been created. This process is then repeated so that there are 1,000 simulated datasets.

Armed with these datasets, the researcher can now compare the statistical moments of the actual, observed distribution against the simulation without fearing the overrejection of the nulls. \citet{AbramsJudgesVaryTheir2012} compute the 25-75 percent interquartile range (IQR) for each variable in each simulation to create a random distribution of sentencing against which they compare the empirical IQR. For the case of random assignment, in which control variables should be used, the expected outcome is that the empirical IQR will be indistinguishable from the simulated distribution of random IQR, thus failing to reject the \emph{F}-test on equality of judge fixed-effects. For the court outcome case, the expected outcomes are that the empirical IQR is statistically different than the simulated IQR distribution and thus the \emph{F-}test would reject the equality of judge fixed-effects by plaintiff's politician characteristic (the interaction terms in equation \refp{eq:methods2}).

\subsection{Court Outcome Prediction} \label{subsec:methods2_paper2}

(TBU)

\section{Further Development} \label{sec:conclusion_paper2}

This paper investigates whether there is differential treatment of politicians in small claims courts in the state of São Paulo, Brazil. To my knowledge, it is the first paper to produce clear evidence on judicial bias in favor of politicians. Similar studies in the literature measure judicial political ideology and bias in high-salience cases in which there is a clear upside for the judiciary if it favors agents from other branches of government. This project supplements these previous initiatives by investigating a more nuanced relationship yet undetected in other analyses.

Besides the evidence of judicial favoritism of politicians, this project also makes predictions about the potential court outcomes if more politicians were taken to court. Though I only focus on a small set of cases, those filed in São Paulo's special civil tribunals (SCTs), the predictions should serve as a benchmark for the deviations between a \emph{de facto} and a \emph{de jure} independent judiciary. There are clear policy improvements discussed in the political economy literature when the judiciary branch is independent and able to check the power of the executive and the legislative \citep{BalandChapter69Governance2010}.

The longest task in this paper is data collection. As discussed in section \ref{sec:data_paper2}, I have a clear roadmap of the next steps necessary for the completion of this project. I will write the web scraping scripts based on existing programs and extract case information once court documents are downloaded. Once data collection is finished, the analysis is fairly straightforward as all steps for measuring the bias effect and machine learning predictions are respectively documented in \citet{AbramsJudgesVaryTheir2012} and \citet{AtheypredictionUsingbig2017,AtheyImpactMachineLearning2019}. Therefore, I believe this project is feasible and it will be completed as a chapter of my dissertation project.


\clearpage

\setlength\bibsep{0pt}
\bibliographystyle{apalike}
\bibliography{/Users/aassumpcao/library.bib}

\end{document}