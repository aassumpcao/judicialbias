\documentclass[11pt]{article}
\usepackage{amssymb}
\usepackage{amsmath}
\usepackage{centernot}
\usepackage{amsfonts}
\usepackage{eurosym}
\usepackage{geometry}
\usepackage{ulem}
\usepackage{graphicx}
\usepackage{tikz}
\usepackage{rotating}
\usepackage{caption}
\usepackage{color}
\usepackage{setspace}
\usepackage{sectsty}
\usepackage{comment}
\usepackage{footmisc}
\usepackage[inline]{enumitem}
\usepackage{caption}
\usepackage{natbib}
\usepackage{pdflscape}
\usepackage{subfigure}
\usepackage{array}
\usepackage{titling}
\usepackage{multirow}
\usepackage{diagbox}
\usepackage{dcolumn}
\usepackage{makecell}
\usepackage[hidelinks]{hyperref}
\hypersetup{unicode = true}

\normalem

\onehalfspacing
\newtheorem{theorem}{Theorem}
\newtheorem{corollary}[theorem]{Corollary}
\newtheorem{proposition}{Proposition}
\newenvironment{proof}[1][Proof]{\noindent\textbf{#1.}}{\ \rule{0.5em}{0.5em}}

\newtheorem{hyp}{Hypothesis}
\newtheorem{subhyp}{Hypothesis}[hyp]
\renewcommand{\thesubhyp}{\thehyp\alph{subhyp}}
% \renewcommand{\labelenumi}{H.\arabic{enumi}.} % Redefine new labels for hyp
\newcommand{\T}{\rule{0pt}{2.6ex}}            % Top strut
\newcommand{\B}{\rule[-1.2ex]{0pt}{0pt}}      % Bottom strut
\newcommand{\red}[1]{{\color{red} #1}}
\newcommand{\blue}[1]{{\color{blue} #1}}
\newcommand{\ci}{\perp\!\!\!\perp}
\newcommand{\nci}{\centernot{\ci}}
\newcommand{\subtitle}[1]{\posttitle{\par\end{center}\begin{center}\large#1\end{center}\vskip0.5em}}
\newcommand{\refp}[1]{(\ref{#1})}

\newcolumntype{L}[1]{>{\raggedright\let\newline\\arraybackslash\hspace{0pt}}m{#1}}
\newcolumntype{C}[1]{>{\centering\let\newline\\arraybackslash\hspace{0pt}}m{#1}}
\newcolumntype{R}[1]{>{\raggedleft\let\newline\\arraybackslash\hspace{0pt}}m{#1}}

\geometry{left=1.0in,right=1.0in,top=1.0in,bottom=1.0in}

\begin{document}

\begin{titlepage}
\title{Judicial Favoritism of Politicians: Evidence from Small Court Claims}
\author{Andre Assumpcao\thanks{PhD Student, Department of Public Policy, The University of North Carolina at Chapel Hill. Contact details: \href{mailto:aassumpcao@unc.edu}{\textcolor{blue}{aassumpcao@unc.edu}}}}
\date{December 21, 2018}

\maketitle

\begin{abstract}
\noindent TBU. \\
\vspace{0in} \\
\noindent\textbf{Keywords:} political economy of development; judicial politics; judicial bias. \\

\noindent\textbf{JEL classification:} D73; K42; P48; H83. \\

\vspace{0in}
\bigskip

\end{abstract}

\setcounter{page}{0}

\thispagestyle{empty}

\end{titlepage}

\clearpage

\section{Introduction} \label{sec:introduction_paper2}

Suppose a court case involving a politician is brought before an independent, high-quality judicial system. Judges make their decision uniquely based on case merits. Assume further that lawyer skills and case merits are randomly distributed across plaintiffs and defendants. Under these simplifying conditions, politicians should not expect a higher win rate at trial than ordinary citizens. Surprisingly, however, there is not much evidence on judicial independence in cases involving politicians. This paper is one of the first attempts at measuring judicial impartially and predicting court outcomes when politicians are before judges.

There is comprehensive evidence for other discrimination effects. \citet{ShayoJudicialIngroupBias2011} document a positive in-group bias, or the preferential treatment, of 17 to 20 percentage points when judge and litigants have the same ethnicity. \citet{AbramsJudgesVaryTheir2012} find that African American defendants are 18 percentage points more likely to be incarcerated than white defendants. \citet{LuPoliticalConnectednessCourt2015} show that politically connected firms are more likely to have favorable judicial rulings in property rights cases. There are a number of additional cases reported in \citet{RachlinskiJudgingJudiciaryNumbers2017}.

Isolating the effect of judicial favoritism on court outcomes is not easy. Court cases are filled with sources of heterogeneity. Judges, plaintiffs, and claimants have individual traits that could influence a court outcome, such as their gender, ethnicity, religion, wealth, and so on. Litigants might also have access to heterogeneous pools of lawyers; any minimal discrepancy in skills might be the deciding factor driving the outcome of a case. Finally, case circumstances and merits change substantially and can determine how any single judge will rule. For instance, \citet{LimJudgePoliticianPress2015} evaluate whether judicial decisions are influenced by media coverage and find that nonpartisan U.S. State Court judges increases sentence length in violent crimes by 3.4 percent (equivalent to six months of extra jail time). The combination of these factors make it hard for the isolation of a single effect on court outcomes.

In recent years, however, there has been growing interest in the relationship between judicial decisions and politics. Using data on employment claims filed in Venezuela between 2006 and 2017, \citet{Sanchez-MartinezDismantlingInstitutionsCourt2018} looks at whether defendant employers are more likely to see a favorable outcome when they are affiliated with the United Socialist Party of Venezuela (PSUV), in power since 2007. The author finds that employers who share the same party affiliation as judges are 20 percent more successful at trial. In \citet{LambaisJudicialSubversionEvidence2018}, the authors identify a 50 percentage point advantage in the win rate at court for elected versus non-elected candidates when both are defendants in corruption cases filed only after election day has passed. This project supplements the recent literature by investigating whether favoritism persists in cases where judges and politicians have less at stake (e.g.~small claim cases), testing a theory of personal ties across members of each branch of government.

Besides the theoretical component, this paper makes a series of data analyses contributions to the literature in law and politics. In order to measure the effect of political bias, I scrape and code judicial decisions in the São Paulo State Court System (Brazil) for all elected mayors and city councilors since 2012. I then apply the methodology in \citet{AbramsJudgesVaryTheir2012} to evaluate whether politicians have an upper hand in small claims cases. I construct a random distribution of court outcomes against which I compare the observed outcomes in the data scraped from the web. Finally, I employ text analysis to extract additional information from judicial decisions and construct supervised machine learning (ML) predictions of court outcomes based on case and politician characteristics for the entire distribution of elected politicians in the State of São Paulo between 2008 and 2018. The web scraper, the simulations, and the ML algorithms are made available as free software for researchers conducting similar research projects using judiciary data in other countries.

The remainder of this paper is as follows. Section \ref{sec:background_paper2} presents the institutional environment of Brazil's State Court System. Section \ref{sec:data_paper2} summarizes the test dataset used for analysis; section \ref{sec:methods_paper2} discusses the analytical strategy and, finally, section \ref{sec:conclusion_paper2} lays out the necessary steps for completion of this project.

\section{Institutional Background} \label{sec:background_paper2}

Brazil's judiciary system is divided into general and limited jurisdiction courts. Federal and State Courts from the general system and Electoral, Military, and Labor Courts form the limited jurisdiction system. There are three instances of judicial review in either system and the court of last resort is the Federal Supreme Court (STF). It takes up cases under its jurisdiction as set out in the Brazilian Constitution, cases in which there are conflicting norms or jurisprudence issued by lower courts, and cases where there is a direct violation of constitutional norms. To limit the sources of heterogeneity, this paper focuses on cases heard at state court systems. In particular, I focus on the state of São Paulo, the most economically and politically important state in the country.

There are 319 judicial districts in the state and each district has one or more courthouses. These courthouses host at least one judge with either broad attribution, meaning that they can rule on any issue within the state court system jurisdiction, or specialized attribution, which means they only oversee certain types of cases within the system, e.g.~commercial or family law. Within the state system, there are specialized small claims courts called \emph{Juízados Especiais Cíveis} (Special Civil Tribunals, in free translation, and SCTs henceforth). SCTs replaced the primary small claims courts across Brazil upon the passage of the latest Brazilian Constitution in 1988.\footnote{More evidence of this in \citet{LichandAccessJusticeEntrepreneurship2014}.} Their goal is to simplify and increase access to justice across states by means of removing many procedural requirements present in the other litigation instruments. SCTs are the primary judicial body for small complexity cases, defined as cases in which claims do not exceed 40 times the minimum wage\footnote{There are no state minimum wages in Brazil, so this is the federal, natiowide minimum wage at R\$ 954.00 in 2018. This is equivalent to ~\$10,500 in current dollars using the 2018 exchange rate average} involving breach of lease contract, consumer rights, debt execution, tort, and so on. There is no need for an attorney if claims are under 20 times the minimum wage. SCTs are only open to individual or small company plaintiffs.

An example helps illustrate a typical SCT case. Suppose your mobile phone service provider has been overbilling you for international phone calls that were never made. You, unfortunately, could not resolve this issue with the company's customer service and now would like to take legal action and receive financial compensation for the wrongful charges to your bill. You walk up to an SCT office, speak to a courthouse clerk and file your claim along with any supporting documentation. The clerk then provides a court date for a conciliation hearing. At the hearing, you and the phone company will try to reach an agreement; if that fails, the judge sets trial for either later that same day or in the following days. At trial, the judge issues a sentence which can be appealed within 10 days; on appeal, a three-judge panel then issues the final ruling. This entire process might take less than three months, representing a substantial improvement when compared to cases in the regular judicial process at other state courts.

SCT structure greatly reduces the number of dimensions driving judicial decisions. According to the São Paulo State Court website, there are less than 15 types of cases that can be brought before SCTs. It is easier for judges and lawyers, when hired by the parts, to specialize and reduce any skill discrepancy that could substantially alter a case outcome. In addition, the sentence is standard across cases: the losing side will pay the claim amount to the winning side, which is capped at 40x the minimum wage. The standard, and relatively low salience, punishment to litigants removes an additional source of heterogeneity from high-profile cases, such as corruption cases in \citet{LambaisJudicialSubversionEvidence2018} or violent crimes in \citet{LimJudgePoliticianPress2015}. In fact, the use of small claims court is an approach first introduced by \citet{ShayoJudicialIngroupBias2011}, that take advantage of the relative homogeneity of small claims cases in Israel to isolate the effect of ethnicity on court outcomes. Lastly, judges have no control over which SCT cases they take. In single-judge benches, all cases are presented before the same judge; in multiple-judge benches, the cases are randomly distributed to judges assigned to each SCT.\footnote{There are exceptions when (...).} These distribution rules are again dimension-reducing and prevent that cases are differently distributed to systematically more lenient (or harsher) judges at the state system. Yet, for robustness purposes, I replicate the process in \citet{AbramsJudgesVaryTheir2012} producing random distributions of court outcomes to serve as a check on the quality of the case assignment system implemented by the state of São Paulo.


\section{Data} \label{sec:data_paper2}


\begin{table}[!htbp] \centering
  \caption{Descriptive Statistics}
  \label{tab:sumstats}
\scriptsize
\begin{tabular}{@{\extracolsep{5pt}}lccccc}
\\[-1.8ex]\hline
\hline \\[-1.8ex]
& \multicolumn{1}{c}{N} & \multicolumn{1}{c}{Mean} & \multicolumn{1}{c}{St. Dev.} & \multicolumn{1}{c}{Min} & \multicolumn{1}{c}{Max} \T \B \\
\hline \\[-1.8ex]
Age                             & 15,232 & 45.384 & 10.638 & 18 & 89 \\
Male                            & 15,232 & .882   & .322   & 0  & 1  \\
Political Experience            & 15,232 & .139   & .346   & 0  & 1  \\
Campaign Expenditures (ln)      & 15,232 & 9.232  & 4.099  & 0  & 16 \\
% Politician is Plaintiff         & 15,232 & .504   & .500   & 0  & 1  \\
% Probability of Favorable Ruling & 15,232 & .509   & .500   & 0  & 1  \\
\\[-1.8ex]\hline
\hline \\[-1.8ex]
\end{tabular}
\end{table}


\section{Analytical Strategy} \label{sec:methods_paper2}

\section{Further Development} \label{sec:conclusion_paper2}

\clearpage

\setlength\bibsep{0pt}
\bibliographystyle{apalike}
\bibliography{/Users/aassumpcao/library.bib}

\end{document}