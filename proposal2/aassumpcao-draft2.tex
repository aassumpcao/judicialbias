\documentclass[11pt]{article}
\usepackage{amssymb}
\usepackage{amsmath}
\usepackage{centernot}
\usepackage{amsfonts}
\usepackage{eurosym}
\usepackage{geometry}
\usepackage{ulem}
\usepackage{graphicx}
\usepackage{tikz}
\usepackage{rotating}
\usepackage{caption}
\usepackage{color}
\usepackage{setspace}
\usepackage{sectsty}
\usepackage{comment}
\usepackage{footmisc}
\usepackage[inline]{enumitem}
\usepackage{caption}
\usepackage{natbib}
\usepackage{pdflscape}
\usepackage{subfigure}
\usepackage{array}
\usepackage{titling}
\usepackage{multirow}
\usepackage{diagbox}
\usepackage{dcolumn}
\usepackage{makecell}
\usepackage[hidelinks]{hyperref}
\hypersetup{unicode = true}

\normalem

\onehalfspacing
\newtheorem{theorem}{Theorem}
\newtheorem{corollary}[theorem]{Corollary}
\newtheorem{proposition}{Proposition}
\newenvironment{proof}[1][Proof]{\noindent\textbf{#1.}}{\ \rule{0.5em}{0.5em}}

\newtheorem{hyp}{Hypothesis}
\newtheorem{subhyp}{Hypothesis}[hyp]
\renewcommand{\thesubhyp}{\thehyp\alph{subhyp}}
% \renewcommand{\labelenumi}{H.\arabic{enumi}.} % Redefine new labels for hyp
\newcommand{\T}{\rule{0pt}{2.6ex}}            % Top strut
\newcommand{\B}{\rule[-1.2ex]{0pt}{0pt}}      % Bottom strut
\newcommand{\red}[1]{{\color{red} #1}}
\newcommand{\blue}[1]{{\color{blue} #1}}
\newcommand{\ci}{\perp\!\!\!\perp}
\newcommand{\nci}{\centernot{\ci}}
\newcommand{\subtitle}[1]{\posttitle{\par\end{center}\begin{center}\large#1\end{center}\vskip0.5em}}
\newcommand{\refp}[1]{(\ref{#1})}

\newcolumntype{L}[1]{>{\raggedright\let\newline\\arraybackslash\hspace{0pt}}m{#1}}
\newcolumntype{C}[1]{>{\centering\let\newline\\arraybackslash\hspace{0pt}}m{#1}}
\newcolumntype{R}[1]{>{\raggedleft\let\newline\\arraybackslash\hspace{0pt}}m{#1}}

\geometry{left=1.0in,right=1.0in,top=1.0in,bottom=1.0in}

\begin{document}

\begin{titlepage}
\title{Judicial Favoritism of Politicians: Evidence from Small Court Claims}
\author{Andre Assumpcao\thanks{PhD Student, Department of Public Policy, The University of North Carolina at Chapel Hill. Contact details: \href{mailto:aassumpcao@unc.edu}{\textcolor{blue}{aassumpcao@unc.edu}}}}
\date{December 21, 2018}

\maketitle

\begin{abstract}
\noindent TBU. \\
\vspace{0in} \\
\noindent\textbf{Keywords:} political economy of development; judicial politics; judicial bias. \\

\noindent\textbf{JEL classification:} D73; K42; P48; H83. \\

\vspace{0in}
\bigskip

\end{abstract}

\setcounter{page}{0}

\thispagestyle{empty}

\end{titlepage}

\clearpage

\section{Introduction} \label{sec:introduction_paper2}

Suppose a court case involving a politician is brought before an independent, high-quality judicial system. Judges make their decision uniquely based on case merits. Assume further that lawyer skills and case merits are randomly distributed across plaintiffs and defendants. Under these simplifying conditions, politicians should not expect a higher win rate at trial than ordinary citizens. Surprisingly, however, there is not much evidence on judicial independence in cases involving politicians. This paper is one of the first attempts at measuring judicial impartially and predicting court outcomes when politicians are before judges.

There is comprehensive evidence for other discrimination effects. \citet{ShayoJudicialIngroupBias2011} document a positive in-group bias, or the preferential treatment, of 17 to 20 percentage points when judge and litigants have the same ethnicity. \citet{AbramsJudgesVaryTheir2012} find that African American defendants are 18 percentage points more likely to be incarcerated than white defendants. \citet{LuPoliticalConnectednessCourt2015} show that politically connected firms are more likely to have favorable judicial rulings in property rights cases. There are a number of additional cases reported in \citet{RachlinskiJudgingJudiciaryNumbers2017}.

Judicial favoritism is not easily identifiable. Court cases are filled with sources of heterogeneity. Judges, plaintiffs, and claimants have individual traits that could influence a court outcome, such as their gender, ethnicity, religion, wealth, and so on. Second, litigants might also have access to heterogeneous pools of lawyers; any minimal discrepancy in skills might be the deciding factor driving the outcome of a case. Third, case circumstances and merits change substantially and can determine how any single judge will rule. \citet{LimJudgePoliticianPress2015} evaluate whether judicial decisions are influenced by media coverage and find that nonpartisan U.S. State Court judges increases sentence length in violent crimes by 3.4 percent (equivalent to six months of extra jail time).

\section{Institutional Background} \label{sec:background_paper2}

\section{Data} \label{sec:data_paper2}


\begin{table}[!htbp] \centering
  \caption{Descriptive Statistics}
  \label{tab:sumstats}
\scriptsize
\begin{tabular}{@{\extracolsep{5pt}}lccccc}
\\[-1.8ex]\hline
\hline \\[-1.8ex]
& \multicolumn{1}{c}{N} & \multicolumn{1}{c}{Mean} & \multicolumn{1}{c}{St. Dev.} & \multicolumn{1}{c}{Min} & \multicolumn{1}{c}{Max} \T \B \\
\hline \\[-1.8ex]
Age                             & 15,232 & 45.384 & 10.638 & 18 & 89 \\
Male                            & 15,232 & .882   & .322   & 0  & 1  \\
Political Experience            & 15,232 & .139   & .346   & 0  & 1  \\
Campaign Expenditures (ln)      & 15,232 & 9.232  & 4.099  & 0  & 16 \\
% Politician is Plaintiff         & 15,232 & .504   & .500   & 0  & 1  \\
% Probability of Favorable Ruling & 15,232 & .509   & .500   & 0  & 1  \\
\\[-1.8ex]\hline
\hline \\[-1.8ex]
\end{tabular}
\end{table}


\section{Empirical Strategy} \label{sec:methods_paper2}

\section{Preliminary Results} \label{sec:results_paper2}

\section{Further Development} \label{sec:conclusion_paper2}

\clearpage

\setlength\bibsep{0pt}
\bibliographystyle{apalike}
\bibliography{/Users/aassumpcao/library.bib}

\end{document}